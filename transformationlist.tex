\documentclass[10pt]{article}

\usepackage[fleqn]{amsmath}
\usepackage{fullpage}

\begin{document}
\title{List of Transformations for Code and Relational Database Evolution}
\author{Ondrej Macek and Karel Richta}
\maketitle

\tableofcontents \cleardoublepage

\section{Notation}

\begin{itemize}
    \item $X = A*$ means X is defined as a sequence of elements from A,
    \item $X = (A,B)$ means X is a tuple of pairs from A and B,
    \item $X = A | B$ means X is either A or B.
\end{itemize}

\subsection{State}
To distinguish the initial and final state of the transformation we use the symbol ' (apostrophe) to annotate all elements of the final state. Two variables $x$, $x'$ represent the same element of the model variable $x$ in the initial state and $x'$ in the final state. If there are no explicitly defined differences between $x$ and $x'$ then we assume $x = x'$. If $x$ is a tuple and there are defined differences only for part of this tuple, then we assume the rest of the tuple stays unchanged.

\subsection{Transformation Definition}
There is defined only successful processing of a transformation in the following text. All undefined paths results in $\perp$ i.e. in inconsistent state of software.


%%%%%%%%%%%%%%%%%%%%%%%%%%%%%%%%%%%%%%%%%%%%%%%%%%%%%%%
\section{Application Model}
\begin{align}
& \mathbf{AppType} = APPSTRING \;|\; APPINTEGER \;|\; APPBOOLEAN \\
& \mathbf{InheritanceType} = SINGLETABLE \; | \; TABLE-PER-CLASS  \; |  \; \\ & \;\;\;\;   JOINED\\
& \mathbf{Inheritance} = (class, InheritanceType) \cup OBJECT \\
& \mathbf{Association} = (label, classRef, startCardinality, endCardinality)  \\
& \mathbf{Property} = (label, AppType, DefaultValue, \nonumber \\ & \;\;\;\;  Cardinality, Mandatory) \\
& \mathbf{Class} = (label, Property*, Association*, Inheritance) \\
& \mathbf{Application} = (Class*) \\
\end{align}

Only one type of \emph{InheritanceType} -- \emph{SINGLETABLE} -- is used in definitions of transformations in Sec. \ref{sec:transformations} for the sake of abbreviation. 


%%%%%%%%%%%%%%%%%%%%%%%%%%%%%%%%%%%%%%%%%%%%%%%%%%%%%%%
\section{Database Model}
\subsection{Database Schema}
\begin{align}
&	\mathbf{Database} = ( TableSchema*, TableData* )\\
&	\mathbf{TableSchema} = (label, primaryKey, Column*, ForeignKey*)\\
&	\mathbf{Column} = (label, DbType, DefaultValue, Constraint*) \\
&	\mathbf{ForeignKey} = (label, TableSchema, Constraint*) \\
&	\mathbf{PrimaryKey} =  ( label, Sequence ) \\
&	\mathbf{DbType} = DBSTRING \; | \; DBINT \; | \; DBBOOLEAN\\
&	\mathbf{Constraint} = NOTNULL \; | \; UNIQUE 
\end{align}

\subsection{Data}
\begin{align}
&	\mathbf{TableData} = (Table, KeyPair, Pair*) \\
&   \mathbf{KeyPair} = (PrimaryKey, Value) \\
&	\mathbf{Pair} = ColumnValue \; | \; ForeignKeyValue \\
&   \mathbf{ColumnValue} = (Column, Value) \\
&   \mathbf{ForeignKeyValue} = (ForeignKey, Value)
\end{align}

\subsubsection{Mapping Between Instances}
Some transformations affect stored data. A relation between data from different $TableData$ has to be known during execution of some transformations (e.g. $moveProperty$). The relation is defined as a mapping between $TableData$s. The mapping is defined as follows:
\begin{align}
& mapping: TableData \rightarrow TableData \cup \emptyset %TODO use different symbol for empty mapping
\end{align}
The mapping has a set of $TableData$ in its range set, this allows to define one-to-many and many-to-many relations between data. The $\emptyset$ represents a situation where there is no relation for a given element of the mapping's range. A special case of mapping is an empty mapping denoted as $m_e$, which is used when there are no $TableData$ in the domain or the range i.e. the transformation takes part on the structural level only.

Each mapping has to fulfill constraints given by the structural definition of its range $TableData$. Concretely the uniqueness of column values:
\begin{align}
& \forall \; m \in Mapping;  x_1, x_2 \in domain(m); p_1 \in pairs(m(x_1)),  p_2 \in pairs(m(x_2)): \nonumber \\ 
&  x_1 \neq x_2 \land \exists c \in Column, UNIQUE \in constraints(c) \land \nonumber \\ 
&c \in pairs(columns(range(m))) \implies  p_1 \neq p_2
\end{align}
if the principle of uniqueness is violated then usage od such mapping leads to an inconsistent database. Next constraint of mapping is the non emptiness of columns constrained with $NOTNULL$ constraint:
\begin{align}
& \forall \; m \in Mapping;  x \in domain(m) :  \nonumber \\
& \exists \; c \in Column, NOTNULL \in constraints(c) \implies m(x) \neq \emptyset
\end{align}
if this principle is violated then usage of such mapping leads to an inconsistent database. 

There can occur data loss, when the mapping is a partial function. Usage of such mapping has to be reconsidered before its usage, because it can result in a semantically inconsistent state of the database. 

The mapping can be implemented as a relation between two tables. This can be implemented e.g. as an equality of some columns or a reference from one table to the other. However, the real-life situation needs sometimes more difficult mappings, which can be implemented as a view or nested select commands. %The mapping can provide more than one match for a $TableData$ in the mapping's domain. 



%%%%%%%%%%%%%%%%%%%%%%%%%%%%%%%%%%%%%%%%%%%%%%%%%%%%%%%
\section{Software Model}
\begin{align}
& software(a, d, \rho) = \begin{cases}
consistentSoftware(a, d, \rho) \; \textbf{if} \; a \neq \perp \; \wedge \; d \neq \perp \\ \wedge \; \rho(a) = d 
 \\\\
 \perp
 \end{cases}\\
& a \in Application, d \in Database, \rho \in ORM \nonumber
\end{align}



%%%%%%%%%%%%%%%%%%%%%%%%%%%%%%%%%%%%%%%%%%%%%%%%%%%%%%%
\section{Transformations}
\label{sec:transformations}
\subsection{Application Manipulation}
\subsubsection{newApplication}
Creates a new application, which does not contain classes.
\begin{align}
newApplication \rightarrow Application
\end{align}

\noindent \textbf{Semantics}:
\begin{align}
newApplication() = a \implies classes(a) = \emptyset
\end{align}

\subsubsection{addClass}
Inserts a class into the existing application.
\begin{align}
addClass: Class \times Application \rightarrow Application
\end{align}

\noindent \textbf{Semantics}:
\begin{align}
& addClass(c, a) = a' \implies classes(a') = classes(a) \cup c \nonumber \\
& \;\;\; \mathbf{if} \;\; \forall c_a \in classes(a): label(c_a) \neq label(c)
\end{align}

\subsubsection{addProperty}
Inserts a property into the given class in the application. Overriding of properties in inheritance hierarchy is prohibited.
\begin{align}
addProperty: Class \times Property \times Application \rightarrow Application 
\end{align}

\noindent \textbf{Semantics}:
\begin{align}
& addProperty(c, p, a) = a' \implies properties(c') = properties(c) \cup \{ p \}  \land \nonumber \\
& \;\;\; classes(a') = classes(a) \setminus c  \cup \{c'\}  \nonumber \\
& \;\;\; \mathbf{if} \;\; \forall p_c \in properties(c) : label(p_c) \neq label(p) \land c \in classes(a) \land \; !\:propertyInParent(p, c)
\end{align}

\subsubsection{addAssociation}
Inserts an association between two classes existing in the application.
\begin{align}
addAssociation: Class \times Association  \times Application \rightarrow Application
\end{align}

\noindent \textbf{Semantics}:
\begin{align}
& addAssociation(c, as, a) = a' \implies associations(c') = associations(c) \cup \{ as \}  \land \nonumber \\
& \;\;\; classes(a') = classes(a) \setminus c  \cup \{c'\}  \nonumber \\
& \;\;\; \mathbf{if} \;\; \forall as_c \in associations(c) : label(as_c) \neq label(as) \land \nonumber \\
& \;\;\;\;\;\;\;\; reference(as) \in classes(a) \land c \in classes(a) \land \; !\: associationInParent(a, c)
\end{align}

\subsubsection{renameProperty}
Renames a property in the class.
\begin{align}
renameProperty: Class \times Property \times Label \times Application \rightarrow Application
\end{align}

\noindent \textbf{Semantics}:
\begin{align}
& renameProperty(c, p, l, a) = a' \implies  label(p') = l  \land properties(c') = properties(c) \setminus p \cup \{ p' \} \land \nonumber \\
& \;\;\; classes(a') = classes(a) \setminus c  \cup \{c'\} \land \; ! \: propertyInParent(p', c') \nonumber \\
& \;\;\; \mathbf{if} \;\; \forall p_c \in properties(c) \land p_c \neq p : label(p_c) \neq l \land c \in classes(a) \land p \in properties(c)
\end{align}

\subsubsection{renameAssociation}
Renames an association in the class.
\begin{align}
renameAssociation: Class \times Association \times Label \times Application \rightarrow Application
\end{align}

\noindent \textbf{Semantics}:
\begin{align}
& renameAssociation(c, as, l, a) = a' \implies  label(as') = l  \land land \nonumber \\
& \;\;\; associations(c') = associations(c) \setminus as \cup \{ as' \} \land classes(a') = classes(a) \setminus c  \cup \{c'\} \land \nonumber \\
& \;\;\; \; ! \: associationInParent(a', c') \nonumber \\
& \;\;\; \mathbf{if} \;\; \forall as_c \in associations(c) \land as_c \neq as : label(as_c) \neq l \land c \in classes(a) \land as \in associations(c)
\end{align}

\subsubsection{renameClass}
Renames a class in the application.
\begin{align}
renameClass: Class \times Label \times Application \rightarrow Application
\end{align}

\noindent \textbf{Semantics}:
\begin{align}
& renameClass(c, l, a) = a' \implies label(c') = l \land classes(a') = classes(a) \setminus c  \cup \{c'\} \nonumber \\
& \;\;\; \mathbf{if} \;\; \forall c_a \in classes(a) \land c_a \neq c: label(c_a) \neq l  \land c \in classes(a)
\end{align}

\subsubsection{removeProperty}
Removes property from the class and the application.
\begin{align}
removeProperty: Class \times Property \times Application \rightarrow Application
\end{align}

\noindent \textbf{Semantics}:
\begin{align}
& removeProperty(c, p, a) = a' \implies properties(c') = properties(c) \setminus p   \land \nonumber \\
& \;\;\; classes(a') = classes(a) \setminus c  \cup \{c'\}  \nonumber \\
& \;\;\; \mathbf{if} \;\; p \in properties(c) \land c \in classes(a)
\end{align}

\subsubsection{removeAssociation}
Removes an association between two classes.
\begin{align}
removeAssociation: Class \times Association \times Application \rightarrow Application
\end{align}

\noindent \textbf{Semantics}:
\begin{align}
& removeAssociation(c, as, a) = a' \implies associations(c') = associations(c) \setminus as \land \nonumber \\
& \;\;\; classes(a') = classes(a) \setminus c  \cup \{c'\} \nonumber \\
& \;\;\; \mathbf{if} \;\; c \in classes(a) \land as \in associations(c)
\end{align}

\subsubsection{removeClass}
Removes a class from the application.
\begin{align}
removeClass: Class \times Application \rightarrow Application
\end{align}

\noindent \textbf{Semantics}:
\begin{align}
& removeClass(c, a) = a' \implies  classes(a') = classes(a) \setminus c \nonumber \\
& \;\;\; \mathbf{if} \;\; c \in classes(a) \land \; ! \: isReferenced(c, a) \land \; ! \: isParent(c, a)
\end{align}

\subsubsection{addParent}
Creates a child -- parent relationship between two classes.
\begin{align}
addParent: Class \times Inheritance \times Application \rightarrow Application \\
\end{align}

\noindent \textbf{Semantics}:
\begin{align}
& addParent(c, i, a) = a' \implies parent(c') = i  \nonumber \\
& \;\;\; classes(a') = classes(a) \setminus c  \cup \{c'\} \nonumber \\
& \;\;\; \mathbf{if} \;\; c \in classes(a) \land class(i) \in classes(a) \land parent(c)= OBJECT \land \nonumber \\
& \;\;\; ! \: selfParentPossibility(c, class(i), a)
\end{align}

\subsubsection{removeParent}
Destroys a child -- parent relationship between two classes.
\begin{align}
removeParent: Class \times Application \rightarrow Application \\
\end{align}

\noindent \textbf{Semantics}:
\begin{align}
& removeParent(c, a) = a' \implies parent(c') = OBJECT \nonumber \\
& \;\;\; \mathbf{if} \;\; c \in classes(a)
\end{align}

\subsubsection{pushDown}
Moves the selected property from parent to all its child classes. 
\begin{align}
pushDown: Class \times Property \times Application \rightarrow Application
\end{align}

\noindent \textbf{Semantics}:
\begin{align}
& pushDown(c, p, a) = a' \implies properties(c') = properties(c) \setminus p \land \nonumber \\ 
& \;\;\; \forall c_{ch} \in classes(a) \land class(parent(c_{ch})) = c : properties(c'_{ch}) = properties(c_{ch}) \cup p \nonumber \\
& \;\;\;\land c'_{ch} \in classes(a') \land c_{ch} \notin classes(a') \land \nonumber \\
& \;\;\; \forall \; p' \in c_{ch} : label(p') \neq p \nonumber \\
& \;\;\; \mathbf{if}  \;\; c \in classes(a) \land \exists c_{ch} \in classes(a) : class(parent(c_{ch})) = c
\end{align}


\subsubsection{pullUp}
Moves the selected property from child to its parent.
\begin{align}
pullUp: Class \times Property \times Application \rightarrow Application
\end{align}
\noindent \textbf{Semantics}:
\begin{align}
& pullUp(c, p, a) = a' \implies \exists e = class(inheritance(c)) : properties(e') = properties(e) \cup p \land \nonumber \\
& \;\;\;  properties(c') = properties(c) \setminus p \nonumber \\
& \;\;\; inheritance(c) \neq OBJECT \land p \in properties(c) \land \nonumber \\
& \;\;\; \forall f \in childern(e,a) \setminus c, q \in properties(f) : label(q) \neq label(p) 
\end{align}

\subsection{Database Schema Manipulation}
The section presents all transformations related with the database schema and data.
\subsubsection{newDatabase}
Creates a new empty database
\begin{align}
newDatabase: \rightarrow Database 
\end{align}

\noindent \textbf{Semantics}:
\begin{align}
newDatabase() = d \implies tableSchema(d) = \emptyset \land tableData(d) = \emptyset
\end{align}

\subsubsection{addTable}
Adds a table schema into the database.
\begin{align}
addTable: TableSchema \times Database \rightarrow Database
\end{align}

\noindent \textbf{Semantics}:
\begin{align}
& addTable(ts, d) = d' \implies tableSchemas(d') = tableSchemas(d) \cup ts \nonumber \\
& \;\;\; \mathbf{if}  \;\; \forall ts_d \in tableSchemas(d) : label(ts) \neq label(ts_d)
\end{align}

\subsubsection{addColumn}
Adds a column into the table schema.
\begin{align}
addColumn: TableSchema \times Column \times Database \rightarrow Database
\end{align}

\noindent \textbf{Semantics}:
\begin{align}
& addColumn(ts, col, d) = d' \implies ts'.columns = ts.columns \cup col \land \nonumber \\
& \;\;\; tableSchemas(d') = tableSchemas(d) \setminus ts \cup ts' \nonumber \\
& \;\;\; \mathbf{if}  \;\;  \forall col_{ts} \in ts.columns : label(col_{ts}) \neq label(col) \land ts \in tableSchemas(d)
\end{align}

\subsubsection{addForeignKey}
Adds a foreign key into the table.
\begin{align}
addForeignKey: TableSchema \times ForeignKey \times Mapping \times Database \rightarrow Database
\end{align}

\noindent \textbf{Semantics}:
\begin{align}
& addForeignKey(ts, fk, m, d) = d' \implies ts'.foreigKeys = ts.foreignKeys \cup fk \land \nonumber \\
& \;\;\; tableSchemas(d') = tableSchemas(d) \setminus ts \cup ts' \land \nonumber \\
% mapování 1:1 a 1:N vkládám do "pravé" tedy k :1 nebo :N
& \;\;\; \forall p_i \in selectAll(ts, d), m_i \in m : p_i \in dom(m_i) \implies insertValue(id(p_i), (fk, id(m(p_i))), d')  \nonumber \\
& \;\;\; \mathbf{if}  \;\;  \forall fk_{ts} \in ts.foreignKeys : label(fk) \neq label(fk_{ts}) \land ts \in tableSchemas(d)
\end{align}

\subsubsection{alterColumnName}
Changes the name of a column in the table schema.
\begin{align}
alterColumnName: TableSchema \times Column \times Label \times Database \rightarrow Database
\end{align}

\noindent \textbf{Semantics}:
\begin{align}
& alterColumnName(ts, col, l, d) = d' \implies label(col') = l \land \nonumber \\
& \;\;\; columns(ts') = columns(t) \setminus col \cup col' \land tableSchemas(d') = tableSchemas(d) \setminus ts \cup ts'  \nonumber \\
& \;\;\; \mathbf{if}  \;\;  \forall col_{ts} \in ts.columns : label(col_{ts}) \neq l \land ts \in tableSchemas(d) \land col \in columns(ts) 
\end{align}

\subsubsection{alterForeignKeyName}
Changes the name of a foreign key in the table schema.
\begin{align}
alterForeignKeyName: TableSchema \times ForeignKey \times Label \times Database \rightarrow Database
\end{align}

\noindent \textbf{Semantics}:
\begin{align}
& alterForeignKeyName(ts, fk, l, d) = d' \implies \implies label(ts') = l \land \nonumber \\
& \;\;\; foreignKeys(ts') = foreignKeys(t) \setminus col \cup col' \land tableSchemas(d') = tableSchemas(d) \setminus ts \cup ts'  \nonumber \\
& \;\;\; \mathbf{if}  \;\;  \forall fk_{ts} \in ts.foreignKeys : label(fk_{ts}) \neq l \land ts \in tableSchemas(d) \land \nonumber \\
& \;\;\;\;\;\; fk \in foreignKeys(ts) 
\end{align}

\subsubsection{alterTableName}
Changes the name of a table in the database schema.
\begin{align}
alterTableName: TableSchema \times Label \times Database \rightarrow Database
\end{align}

\noindent \textbf{Semantics}:
\begin{align}
& alterTableName(ts, l, d) = d' \implies label(ts') = l \land tableSchemas(d') = tableSchemas(d) \setminus ts \cup ts'  \nonumber \\
& \;\;\; \mathbf{if}  \;\;  \forall ts_d \in tableSchemas(d) : label(ts_d) \neq l \land \land ts \in tableSchemas(d) 
\end{align}

\subsubsection{dropColumn}
Removes a column from the table schema.
\begin{align}
dropColumn: TableSchema \times Column \times Database \rightarrow Database
\end{align}

\noindent \textbf{Semantics}:
\begin{align}
& dropColumn(ts, col, d) = d' \implies \\
& \begin{cases}
 columns(ts') = columns(ts) \setminus col \land \nonumber \\
     tableSchemas(d') = tableSchemas(d) \setminus ts \cup ts'  \nonumber \\
 \mathbf{if}  \; col \in columns(ts) \land |col| = 1
 \\\\
 dropColumn(ts, tail(col), dropColumn(ts, head(col), d)) \\
 \mathbf{if} \; |col| > 1
 \end{cases}
\end{align}

\subsubsection{dropEmptyColumn}
Removes a column from the table schema only if there are no data stored.
\begin{align}
dropEmptyColumn: TableSchema \times Column \times Database \rightarrow Database
\end{align}

\noindent \textbf{Semantics}:
\begin{align}
& dropEmptyColumn(ts, col, d) = dropColumn(ts, col, d)  \nonumber \\
& \;\;\; \mathbf{if}  \;\; \forall td \in tableData(d) : table(td) \neq ts \lor \forall cv \in pairs(ts) : column(cv) \neq col
\end{align}


\subsubsection{dropForeignKey}
Removes a foreign key from the table schema.
\begin{align}
dropForeignKey: TableSchema \times ForeignKey \times Database \rightarrow Database
\end{align}

\noindent \textbf{Semantics}:
\begin{align}
& dropForeignKey(ts, fk, d) = d' \implies foreignKeys(ts') = foreignKeys(ts) \setminus fk \land \nonumber \\
& \;\;\ tableSchemas(d') = tableSchemas(d) \setminus ts \cup ts'  \nonumber \\
& \;\;\; \mathbf{if}  \;\; fk \in foreignKeys(ts)
\end{align}

\subsubsection{dropEmptyForeignKey}
Removes a foreign key from the table schema only if there are no data stored.
\begin{align}
dropEmptyForeignKey: TableSchema \times ForeignKey \times Database \rightarrow Database
\end{align}

\noindent \textbf{Semantics}:
\begin{align}
& dropEmptyForeignKey(ts, fk, d) = dropForeignKey(ts, fk, d) \nonumber \\
& \;\;\; \mathbf{if}  \;\;  \forall td \in tableData(d) : table(td) \neq ts \lor \forall fv \in pairs(ts) : foreignKey(fv) \neq fk
\end{align}

\subsubsection{dropTable}
Removes a table schema from the database.
\begin{align}
dropTable: TableSchema \times Database \rightarrow Database
\end{align}

\noindent \textbf{Semantics}:
\begin{align}
& dropTable(ts, d) = d' \implies tableSchemas(d') = tableSchemas(d) \setminus ts  \nonumber \\
& \;\;\; \mathbf{if}  \;\; ts \in tableSchemas(d) \land \; ! \: isReferenced(t, d)
\end{align}

\subsubsection{dropEmptyTable}
Removes a table schema from the database only if there are no stored data.
\begin{align}
dropEmptyTable:  TableSchema \times Database \rightarrow Database
\end{align}

\noindent \textbf{Semantics}:
\begin{align}
& dropEmptyTable(ts, d) = dropTable(ts, d) \nonumber \\
& \;\;\; \mathbf{if} \; \forall td \in tableData(d) : table(td) \neq ts
\end{align}


\subsubsection{copyColumn}
The transformations copies structure of the column from one table schema to another. The data are copied according to the given mapping.
\begin{align}
& copyColumn: TableSchema \times TableSchema \times Column \times Mapping \times Database   \nonumber \\
& \;\;\; \rightarrow Database
\end{align}

\noindent \textbf{Semantics}:
\begin{align}
& copyColumn(ts_1, ts_2, col, m, d) = d' \implies \\ 
& \begin{cases}
 columns(ts_2') = columns(ts_2) \cup col \land \nonumber \\
     \forall p_i \in selectAll(ts_1, d) \land p_i \in dom(m_i), m_i \in m, q_{i} \in ran(m_i) : m(p_i) = q_i \implies \nonumber \\
  insertValue(q_i, (col, valueOfColumn(col, p_i)), d')  \nonumber \\
 \mathbf{if} \; ts_1 \neq ts_2 \land col \in columns(ts_1) \land |col| = 1
 \\\\
 copyColumn(ts_1, ts_2, tail(col), m, copyColumn(ts_1, ts_2, head(col), m, d))\\
 \mathbf{if} \; |col| > 1
 \end{cases}
\end{align}


\subsubsection{copyTable}
The transformations creates a copy of the given table. The new table (copy) has a new name defined by \emph{label}. The data are copied as well.
\begin{align}
copyTable: TableSchema \times Label \times Database \rightarrow Database 
\end{align}
%TODO tady není mapping - musí se vytvořit
\noindent \textbf{Semantics}:
\begin{align}
& copyTable(ts, l m, d) = d' \implies
\exists ts_2 = (l, primaryKey(ts), columns(ts), \nonumber \\
& \;\;\; foreignKeys(ts)) : ts_2 \in tableSchemas(d') \land ts_2 \notin tableSchemas(d) \nonumber \\
& \;\;\; \forall q \in ran(m) : p \in dom(m) \land m(m) = q \land \nonumber \\  
& \;\;\; insertData(next(sequence(primaryKey(ts_2))), pairs(p))  \nonumber \\
& \;\;\; \mathbf{if} \; \forall t \in tableSchemas(d) : label(t) \neq l \land ts \in tableSchemas(d)
\end{align}


%%%%%%%%%%%%%%%%%%%%%%%%%%%%%%%%%%%%%%%%%%%%%%%%%%%%%%%
\section{Evolution}
\subsection{Basic Transformations}

\subsubsection{newSoftware}
Creates a new software with initialized application and database.
\begin{align}
& newSoftware: \rightarrow Software \\
& newSoftware() = s \implies software(\Psi(newSoftware, \emptyset), \Phi(newSoftware, \emptyset))
\end{align}

\noindent \textbf{Interpretation}:
\begin{align}
\Psi(newSoftware, \emptyset) = newApplication()
\end{align}
\begin{align}
\Phi(newSoftware, \emptyset) = newDatabase()
\end{align}


\subsubsection{newClass}
Inserts a class into the application and its image into the database.
\begin{align}
& newClass: Class \times Software \rightarrow Software \\
& newClass(c, s) = s' \implies software(\Psi(newClass(c), application(s)), \nonumber \\
& \;\;\; \Phi(newClass(c), database(s))
\end{align}

\noindent \textbf{Interpretation}:
\begin{align}
\Psi(newClass(c), a) = addClass(c, a)
\end{align}
\begin{align}
\Phi(newClass(c), d) = addTable(ORM(c),  d)
\end{align}


\subsubsection{newProperty}
Inserts a property into the given class in the application and its image into the database.
\begin{align}
& newProperty: Class \times Property \times Software \rightarrow Software \\
& newProperty(c, p, s) = s' \implies \nonumber \\ 
& \;\;\; software(\Psi(newProperty(c, p), application(s)), \Phi(newProperty(c,p), database(s))
\end{align}

\noindent \textbf{Interpretation}:
\begin{align}
\Psi(newProperty(c, p), a) = addProperty(c, p, a)
\end{align}
\begin{align}
\Phi(newProperty(c, p), d) = \begin{cases}
  addColumn(ORM(c), ORM(p),  d) \\ \mathbf{if} \; cardinality(p) = 1  \\\\ 
  \; addTable(ORM(p),  d) \\
  \mathbf{if} \; cardinality(p) > 1  
   \end{cases}
\end{align}


\subsubsection{newAssociation}
Inserts a new association between two existing classes into the application and its image into the database.
\begin{align}
& newAssociation: Class \times Association \times Mapping \times Software \rightarrow Software \\
& newAssociation(c, as, s) = s' \implies \nonumber \\ 
& \;\;\; software(\Psi(newAssociation(c, as), application(s)), \Phi(newAssociation(c, as), database(s))
\end{align}

\noindent \textbf{Interpretation}:
\begin{align}
\Psi(newAssociation(c, as), a) = addAssociation(c, as, a)
\end{align}
\begin{align}
& \Phi(newAssociation(c, as), d) = \forall r \in selectAll(ORM(c), d') : r \in dom(m) \implies \nonumber \\ & \;\;\; insertData(m(r), d') \land d' = addTable(ORM(as),  d) \nonumber \\
& \mathbf{where} \; dom(m) \in selectAll(ORM(c),d) \land ran(m) \in selectAll(ORM(reference(as)), d)
\end{align}


%TODO kolize jmen s aplikační úrovní?
\subsubsection{renameProperty}
The transformation changes the label of the given property.
\begin{align}
& renameProperty: Class \times Property \times Label \times Software \rightarrow Software \nonumber \\
& renameProperty(c, p, l, s) = software(\Psi(renameProperty(c, p, l), application(s)),  \nonumber \\ 
& \;\;\; \Phi(renameProperty(c, p, l), database(s))
\end{align}

\noindent \textbf{Interpretation}:
\begin{align}
\Psi(renameProperty(c, p, l), a) = renamePropery(c, p, l, a) 
\end{align}
\begin{align}
\Phi(renameProperty(c, p, l), d) = \begin{cases}
  alterColumnName(ORM(c), ORM(p), l,  d) \\ \mathbf{if} \; cardinality(p) = 1  \\\\ 
  \; alterTable(ORM(p), l,  d) \\
  \mathbf{if} \; cardinality(p) > 1  
   \end{cases}
\end{align}

%TODO kolize jmen s aplikační úrovní?
\subsubsection{renameAssociation}
The transformation changes the label of the given association.
\begin{align}
& renameAssociation: Class \times Association \times Label \times Software \rightarrow Software \nonumber \\
& renameAssociation(c, as, l, s) = software(\Psi(renameAssociation(c, as, l), application(s)),  \nonumber \\ 
& \;\;\;\Phi(renameAssociation(c, as, l), database(s))
\end{align}

\noindent \textbf{Interpretation}:
\begin{align}
\Psi(renameAssociation(c, as, l), a) = renameAssociation(c, as, l, a)
\end{align}
\begin{align}
\Phi(renameAssociation(c, as, l), d) = alterTableName(ORM(as), l, d)
\end{align}

%TODO kolize jmen s aplikační úrovní?
\subsubsection{renameClass}
The transformation changes the label of the given class.
\begin{align}
& renameClass: Class \times Label \times Software \rightarrow Software \nonumber \\
& renameClass(c, l s) = software(\Psi(renameClass(c, l), application(s)), \nonumber \\
& \;\;\; \Phi(renameClass(c, l), database(s))
\end{align}

\noindent \textbf{Interpretation}:
\begin{align}
\Psi(renameClass(c, l), a) = renameClass(c, l, a)
\end{align}
\begin{align}
\Phi(renameClass(c, l), d) = alterTableName(ORM(c), l, d)
\end{align}

%TODO kolize jmen s aplikační úrovní?
\subsubsection{removeProperty}
The transformation removes the given property from the given class.
\begin{align}
& removeProperty: Class \times Property \times Software \rightarrow Software \nonumber \\
& removeProperty(c, p, s) = \begin{cases}
    software(\Psi(removeProperty(c, p), application(s)),  \\ 
        \;\; \Phi(removeProperty(c, p), database(s)) \\
    \mathbf{if} \; |p| = 1 \\\\
    removeProperty(c, tail(p), removeProperty(c, head(p), s)) \\
        \mathbf{if} \; |p| > 1
 \end{cases}
\end{align}

\noindent \textbf{Interpretation}:
\begin{align}
\Psi(removeProperty(c, p), a) = removeProperty(c, p, a)
\end{align}
\begin{align}
\Phi(removeProperty(c, p), d) = \begin{cases}
 dropColumn(ORM(c), ORM(p), d)) \\ \mathbf{if} \; cardinality(p) = 1  \\\\ 
 dropTable(ORM(p), d) \\ \mathbf{if} \; cardinality(p) > 1 
 \end{cases}
\end{align}

%TODO kolize jmen s aplikační úrovní?
\subsubsection{removeAssociation}
The transformation removes the given association from the software.
\begin{align}
& removeAssociation: Class \times Association \times Software \rightarrow Software \nonumber \\
& removeAssociation(c, as, s) = \begin{cases}
   software(\Psi( removeAssociation(c, as), application(s)), \\ 
    \;\;\; \Phi(removeAssociation(c, as), database(s)) \\
     \mathbf{if} \; |as| = 1 \\\\
   removeAssociation(c, tail(as), removeAssociation(c, head(as), s)) \\
     \mathbf{if} \; |as| > 1 
 \end{cases}
\end{align}

\noindent \textbf{Interpretation}:
\begin{align}
\Psi(removeAssociation(c, as), a) = removeAssociation(c, as, a)
\end{align}
\begin{align}
\Phi(removeAssociation(c, as), d) = dropTable(ORM(as), d)
\end{align}

%TODO kolize jmen s aplikační úrovní?
\subsubsection{removeClass}
The transformation removes the class from the software.
\begin{align}
& removeClass: Class \times Software \rightarrow Software \nonumber \\
& removeClass(c, s) = \begin{cases}
 software(\Psi(removeClass(c), application(s)), \\
\;\;\; \Phi(removeClass(c), database(s)) \\
\mathbf{if} \; |c| = 1 \\\\
 removeClass(tail(c), removeClass(head(c), s)) \\
\mathbf{if} \; |c| > 1
 \end{cases}
\end{align}

\noindent \textbf{Interpretation}:
\begin{align}
\Psi(removeClass(c), a) = removeClass(c, a)
\end{align}
\begin{align}
\Phi(removeClass(c), d) = dropTable(ORM(c), d)
\end{align}

\subsubsection{copyProperty}
The transformation copies the given property from one given class to another.
\begin{align}
& copyProperty: Class \times Class \times Property \times Mapping \times Software \rightarrow Software \\
& copyProperty(c_s, c_t, p, m, s) = software(\Psi(copyProperty(c_s, c_t, m, p), application(s)), \nonumber \\
& \;\;\; \Phi(copyProperty(c_s, c_t, m, p), database(s)))
\end{align}

\noindent \textbf{Interpretation}:
\begin{align}
\Psi(copyProperty(c_s, c_t, p, m), a) = newProperty(c_t, p, a)
\end{align}
\begin{align}
\Phi(copyProperty(c_s, c_t, p, m), d) = \begin{cases}
 copyColumn(ORM(c_s), ORM(c_t), ORM(p), m, d) \\
 \mathbf{if} \; cardinality(p) = 1  \\\\
 copyPropertyAsTable(ORM(c_s), ORM(c_t), ORM(p), m, d) \\ 
 \mathbf{if} \; cardinality(p) > 1
 \end{cases}
\end{align}

\subsubsection{moveProperty}
The transformation moves the given property from one given class to another. A copy of given property is created and the original is then removed.
\begin{align}
& moveProperty: Class \times Class \times Property \times Mapping \times Software \rightarrow Software \\
& moveProperty(c_s, c_t, p, m, s) = \begin{cases}
removeProperty(c_s, p, copyProperty(c_s, c_t, p, m, s)) \\
\mathbf{if} |p| = 1 \\\\
moveProperty(c_s, c_t, tail(p), m, removeProperty(c_s, head(p), \\ \;\;\; copyProperty(c_s, c_t, head(p), m, s))) \\ 
\mathbf{if} |p| > 1
 \end{cases}
\end{align}

\subsubsection{inlineClass}
\begin{align}
& inlineClass:  Class \times Class \times Mapping \times Software \rightarrow Software \\
& inlineClass(c_1, c_2, m, s) = removeClass(c_2, moveProperties(c_1, c_2, m, s)) \\
& \mathbf{where} p =  properties(c_2) \nonumber \\ 
& \;\; moveProperties(c_1, c_2, m, s) = \nonumber \\
& \;\; \;\;\; \forall p \in properties(c2) : moveProperty(c_1, c_2, p, m, s) \nonumber \\
& \mathbf{if} !isReferenced(c_2, application(s))
\end{align}

\subsubsection{mergeClasses}
The transformation merges two classes with the same structure into one class.
\begin{align}
& mergeClasses: Class \times Class \times Label \times Software  \rightarrow Software \\
& mergeClasses(c_1, c_2, l, s) = software(\Psi(mergeClasses(c_1, c_2, l, application(s)), \nonumber \\
& \;\;\; \Phi(mergeClasses(c_1, c_2, l, database(s)) \nonumber \\
& \;\; \mathbf{if} properties(c_1) = properties(c_2) \land \; !isReferenced(c_1, d) \land  \; !isReferenced(c_2, d)
\end{align}
\begin{align}
\Psi(mergeClasses(c_1, c_2, l, a)) = removeClass(c_2, renameClass(c_1, l, a))
\end{align}
\begin{align}
& \Phi(mergeClasses(c_1, c_2, l, d)) = \nonumber \\
& \;\;\; alterTableName(ORM(c_1), l, k(ORM(c_2), selectAll(ORM(c_1), d), d)) \nonumber \\
&  \mathbf{if} \; !\:isReferenced(ORM(c_1), d) \land \; !\:isReferenced(ORM(c_2), d) 
\end{align}
\begin{align}
& k: TableSchema \times TableData \times Database \rightarrow Database \\
& k(ts, td, d) = \begin{cases}
 insertData(ts, tableData(ts, (pk, next(pk)) , pairs(td)) , d) \\
 \mathbf{if} |td| = 1 \\
 \mathbf{where} \; pk = primaryKey(ts) 
 \\\\
 k(ts, tail(td), k(ts, head(td), d))
  \mathbf{if} |td| > 1 \\
 \end{cases}
\end{align}



\subsubsection{splitClass}
The transformation extract a subsequence of properties from the given class into a new class. 
\begin{align}
& splitClass: Class \times Label \times Property \times Software \rightarrow Software \\
& splitClass(c, l, p, s) = software(\Psi(splitClass(c, l, p), application(s)), \nonumber \\
& \;\;\; \Phi(splitClass(c, l, p), database(s)))
\end{align}

\noindent \textbf{Interpretation}:
\begin{align}
& \Psi(splitClass(c, l, p), a) =
removeProperty(c, p, newProperty(c_n, p,  addClass(c_n, a))) \nonumber \\
& \;\;\; c_n = class(l, \emptyset, \emptyset, OBJECT)
\end{align}
\begin{align}
\Phi(splitClass(c, l, p), d) = \begin{cases}
dropColumn(ORM(c), ORM(p), copyColumn(ORM(c), \\  \;\; ORM(propToClass(p)), m, addTable(ORM(propToClass(p, l)), d)))\\
\mathbf{where} \; m = \forall r \in selectAll(ORM(c), d) : \\  \;\; m(r) = tableData((ORM(propToClass(p, l)), keyPair(r), \\  \;\;\;\; pairOfColumn(ORM(p), pairs(r)))) \\ 
  \mathbf{if} \; cardinality(p) = 1   \\\\
alterTableName(ORM(p), l, dropForeignKey(ORM(p), \\ \;\; foreignKeys(ORM(p), d)) \\ 
  \mathbf{if} \; cardinality(p) > 1  
 \end{cases}
\end{align}

%Property To Association
\subsubsection{extractPropertyAsObject}
The transformation changes one property into an object with property containing original value.

\begin{align}
& extractPropertyAsObject: Class \times Property \times Label \times Software \rightarrow Software \nonumber \\
& extractPropertyAsObject(c, p, l, s) = 
 software(\Psi(extractPropertyAsObject(c, p, l), application(s)), \nonumber \\
 & \;\;\; \Phi(extractPropertyAsObject(c, p, l), database(s)))
\end{align}

\noindent \textbf{Interpretation}:
\begin{align}
& \Phi(extractPropertyAsObject(c, p, l), a) =  removeProperty(c, p, addAssociation(c,  \nonumber \\
& \;\;\;\;\;\; association(l, propToClass(p, l), 1,cardinality(p)), \nonumber \\
 & \;\;\;  addClass(propToClass(p, l), a))
\end{align}
\begin{align}
& \Psi(extractPropertyAsObject(c, p, l), d) = \\
& \;\;\;\; \begin{cases}
dropColumn(ORM(c), ORM(p), addForeignKey(ts_t, foreignKey(label(c), ORM(c), NOTNULL), \\
 \;\; copyColumn(ORM(c), ts_t, m, addTable(ts_t, d))) \\
 \mathbf{if}\; cardinality(p) = 1 \\
 \mathbf{where} \; ts_t = ORM(propToClass(p, l)) \land \\
 m \in Mapping, r \in selectAll(ORM(c),d) : m(r) = r 
 \\\\
 alterTableName(ORM(p), l, d) \\
 \mathbf{if}\; cardinality(p) > 1
 \end{cases}
\end{align}


%Association to property
\subsubsection{inlineObjectAsProperty}
The transformation \emph{inlineObjectAsProperty} creates a property from the given class. The constraint is that the class to inline has only one property of primitive type and it is not referenced. The transformation \emph{inlineObjectAsProperty} is the opposite do \emph{extractPropertyAsObject}.
\begin{align}
& inilineObjectAsProperty: Class \times Class \times Mapping \times Software \rightarrow Software \nonumber \\
% má jen jednu property
& inilineObjectAsProperty(c_1, c_2, m, s) = software(\Psi(inilineObjectAsProperty(c_1, c_2, m), \nonumber \\  
& \;\;\; application(s)), \Phi(inilineObjectAsProperty(c_1, c_2, m), database(s)))
\end{align}

\noindent \textbf{Interpretation}:
\begin{align}
\Psi(inilineObjectAsProperty(c_1, c_2, m), a) =  removeClass(c_2, moveProperty(c_1, c_2, p, a))  
\end{align}
\begin{align}
& \Phi(inilineObjectAsProperty(c_1, c_2, m), d) = \nonumber \\ 
& \;\;\; \begin{cases}
 dropTable(ORM(c_2), copyColumn(ORM(c_2), ORM(c_1), ORM(p), m, d) \\
%1:1 
\;\; \mathbf{if} \;  \forall x, z \in dom(m), y \in ran(m) : m(x) = y \land  
 m(z) = y \implies x = z 
 \\\\
 d \\
 %1:N
 \;\; \mathbf{if} \; \exists x \in dom(m), y, z \in ran(m) : m(x) = \{y, z\} \land y \neq z
 \end{cases} \nonumber \\
& \;\;\; \mathbf{if} \; p \in properties(c_2) \land |properties(c_2)| = 1 \land ! isReferenced(c,application(s)) 
\end{align}


%TODO kolize jmen s aplikační úrovní?
\subsubsection{addParent}
Adds a parent - child relationship between the two given classes.
\begin{align}
& addParent: Class \times Inheritance \times Mapping \times Software \rightarrow Software \nonumber \\
& addParent(c, ih, m, s) = software(\Psi(addParent(c, ih, m), application(s)), \nonumber \\
& \;\;\;  \Phi(addParent(c, ih, m), database(s)))
\end{align}

\noindent \textbf{Interpretation}:
\begin{align}
\Psi(addParent(c, ih, m), a) = addParent(c, ih, a)
\end{align}
\begin{align}
\Phi(addParent(c, ih, m), d) = \Phi(mergeClasses(c, class(ih), m), d)
\end{align}

%TODO kolize jmen s aplikační úrovní?
\subsubsection{removeParent}
Destroys the parent-child relationship between two classes. To simplify the model we assume the child class is a leaf of the inheritance hierarchy.
\begin{align}
& removeParent: Class \times Software \rightarrow Software \nonumber \\
& removeParent(c, s) = software(\Psi(removeParent(c), application(s)), \nonumber \\
& \;\;\;  \Phi(removeParent(c), database(s)) \nonumber \\
& \mathbf{if} \; \forall e \in classes(application(a)) : class(inheritance(e)) \neq c
\end{align}

\noindent \textbf{Interpretation}:
\begin{align}
\Psi(removeParent(c), a) = removeParent(c, a)
& \mathbf{if} \; children(c, a) = \emptyset
\end{align}
\begin{align}
&\Phi(removeParent(c), d) = 
%TODO změnit reference
 dropColumn(ORM(class(inheritance(c))), ORM(properties(c)) \setminus \nonumber \\ 
& \;\;\; ORM(properties(class(inheritance(c)))), copyColumn(ORM(class(inheritance(c))), ORM(c), \nonumber \\ 
& \;\;\; ORM(properties(c)), m, 
addTable(ORM(c), d)) \nonumber \\
& \mathbf{where} \; m = \forall r \in selectAll(ORM(c), d) : \nonumber \\  
& \;\; m(r) = tableData((ORM(c), keyPair(r), pairOfColumn(ORM(properties(c)), pairs(r))))  
\end{align}

%TODO kolize jmen s aplikační úrovní?
\subsubsection{pushDown}
The transformations moves one property from the parent class to all child classes.
\begin{align}
& pushDown: Class \times Property \times Software \rightarrow Software \nonumber \\
& pushDown(c, p, s) = software(\Psi(pushDown(c, p), application(s)), \nonumber \\
& \;\;\; \Phi(pushDown(c, p), database(s))
\end{align}

\noindent \textbf{Interpretation}:
\begin{align}
\Psi(pushDown(c, p), a) = pushDown(c, p, a)
\end{align}
\begin{align}
\Phi(pushDown(c, p), d) = d
\end{align}

%TODO kolize jmen s aplikační úrovní?
\subsubsection{pullUp}
The transformation moves one property from the child class into the parent class.
\begin{align}
& pullUp: Class \times Property \times Software \rightarrow Software \nonumber \\
& pullUp(c, p, s) = software(\Psi(pullUp(c, p), application(s)), \Phi(pullUp(c, p), database(s))
\end{align}

\noindent \textbf{Interpretation}:
\begin{align}
\Psi(pullUp(c, p), a) = pullUp(c, p, a)
\end{align}
\begin{align}
\Phi(pullUp(c, p), d) = d
\end{align}

\subsubsection{extractParent}
The transformation creates a new class with the given property and creates the parent-child relationship between the new and the original class.
\begin{align}
& extractParent: Class \times Property \times InheritanceType \times Label \times  \nonumber \\
& \;\;\; ConsistentSoftware \rightarrow Software \\
& extractParent(c, p, it, l, s) = pullUp(c, p, addParent(c,inheritance(e, it), m,  addClass(e, s))) \\
& \mathbf{where} \; e = class(l, \emptyset, \emptyset, OBJECT)  \nonumber \\
& \;\;\; m \in Mapping, dom(m) = selectAll(ORM(c)) : \forall r \in m :  m(r) = r \nonumber
\end{align}


\subsubsection{extractCommonParent}
The transformation creates a new class with the given property, which exists in all given classes and creates the parent-child relationship between the new and all the original classes.
\begin{align}
& extractCommonParent: Class \times Property \times InheritanceType \times Label \times
\nonumber \\
& \;\;\; ConsistentSoftware \rightarrow Software \\
& extractCommonParent(c, p, it, l, s) = \begin{cases}
     extractParent(c, p, it, l, s) \\
     \mathbf{if} \; |c| = 1 \\\\
     h(tail(c), p, it, l, extractParent(head(c), p, it, l, s)) \\
       \mathbf{if} \; |c| > 1
 \end{cases} \\
& \;\;\;\;\;\;\;\;\;\;\;\;\;\;\;\;\;\;\;\;\;\;\;\;\;\;\;\;\;\;\;\;\;\;\;\;\;\;\;\;\;\;\;\;\;\;\;\;\;\;\;\;\;\;\;\;\;\;\;\;\;\;\;\;\; \mathbf{if} \; \forall \: e \in c : p \in properties(e)
\end{align}

\begin{align}
& h: Class \times Property \times InheritanceType \times Label \times
\nonumber \\
& \;\;\; ConsistentSoftware \rightarrow Software \\
& h(c, p, it, l, s) = \begin{cases}
    moveProperty(c, f, p, m, addParent(c, f, s)) \\
    \mathbf{if} \; |c| = 1 \\
   \\
    h(tail(c), p, it, l, moveProperty(head(c), f, p, m, addParent(head(c), f, s)))\\
    \mathbf{if} \; |c| > 1
 \end{cases} \nonumber \\
&   \mathbf{where} \; f = class(l, \emptyset, \emptyset, OBJECT) \land  \\
&    \;\;\; m \in Mapping, dom(m) = selectAll(ORM(c)) : \forall r \in m :  m(r) = r 
\end{align}



%%%%%%%%%%%%%%%%%%%%%%%%%%%%%%%%%%%%%%%%%%%%%%%%%%%%%%%
\section{Helpers}
This section contains functions, which serves to obtain information about a model element.
\subsection{Application Helpers}

\subsubsection{propertyInParent}
Returns true if a property with the same name already exists in some class of the inheritance hierarchy.
\begin{align}
& propertyInParent: Property \times Class \rightarrow Boolean \\
& propertyInParent(p, c) = \begin{cases}
 false \\
 \mathbf{if} \; inheritance(c) = OBJECT \\ \\
 propertyInParent(p, class(inheritance(c))) \\
 \mathbf{if} \; \forall \; p_c \in properties(c) : label(p_c) \neq label(p) \\ \\
 true \\
 \mathbf{if} \; \exists \; p_c \in properties(c) : label(p_c) = label(p) 
 \end{cases}
\end{align}

\subsubsection{associationInParent}
Returns true if a association with the same name already exists in some class of the inheritance hierarchy.
\begin{align}
& associationInParent: Association \times Class \rightarrow Boolean \\
& associationInParent(a, c) = \begin{cases}
 false \\
 \mathbf{if} \; inheritance(c) = OBJECT \\ \\
 associationInParent(a, class(inheritance(c))) \\
 \mathbf{if} \; \forall \; a_c \in associations(c) : label(a_c) \neq label(a) \\ \\
 true \\
 \mathbf{if} \; \exists \; a_c \in associations(c) : label(a_c) = label(a) 
 \end{cases}
\end{align}

\subsubsection{isReferenced}
Returns true if the class is referenced in the given application.
\begin{align}
& isReferenced: Class \times Application \rightarrow Boolean \\
& isReferenced(c, a) = \begin{cases}
 false \\
 \mathbf{if} \; \forall \; x \in classes(a), a_x \in associations(x) : reference(a_x) \neq c \\ \\
 true \\
 \mathbf{if} \; \exists \; x \in classes(a), a_x \in associations(a) : reference(a_x) = c 
 \end{cases}
\end{align}

\subsubsection{isParent}
Returns true if the class is a parent of any other class in the model.
\begin{align}
& isParent: Class \times Application \rightarrow Boolean \\
& isParent(c, a) = \begin{cases}
 false \\
 \mathbf{if} \; \forall \: x \in classes(a) : class(inheritance(x)) \neq c \\\\
 true \\
  \mathbf{if} \; \exists \: x \in classes(a) : class(inheritance(x)) = c
 \end{cases}
\end{align}

\subsubsection{selfParentPossibility}
Returns true if the first given class is in the direct or indirect parent of the second class. 
\begin{align}
& selfParentPossibility: Class \times  Class \rightarrow Boolean \\
& selfParentPossibility(c, c_{parent}) = \begin{cases}
 false \\
 \mathbf{if} \; class(inheritance(c_{parent})) = OBJECT \land c_{parent} \neq c \\\\
 selfParentPossibility(c, class(inheritance(c_{parent}))\\
 \mathbf{if} \; class(inheritance(c_{parent})) \neq c \; \land  \\ \;\; class(inheritance(c_{parent})) \neq OBJECT \\\\
 true \\
 \mathbf{if} \; c_{parent} = c
 \end{cases}
\end{align}

\subsection{Database Helpers}
\subsubsection{selectOne}
Select one \emph{TableData} which references given \emph{TableSchema} and has the given primary key value.
%TODO ID
\begin{align}
selectOne: TableSchema \times ID \times  Database \rightarrow TableData
\end{align}

\textbf{Semantics}:
\begin{align}
& selectOne(ts, id, d) = td \implies td \in tableData(d) \land id(td) = id \land table(td) = ts \nonumber \\
& \;\;\; \mathbf{if}  \;\; ts \in tableSchemas(d)
\end{align}

\subsubsection{selectAll}
Select all \emph{TableData} which references given \emph{TableSchema}.
\begin{align}
selectAll: TableSchema \times  Database \rightarrow TableData
\end{align}

\textbf{Semantics}:
\begin{align}
& selectAll(ts, d) = td \implies td \in tableData(d) \land table(td) = ts \nonumber \\
& \;\;\; \mathbf{if}  \;\; ts \in tableSchemas(d)
\end{align}

\subsubsection{insertData}
Inserts one \emph{TableData} into the database.
\begin{align}
insertData: TableData \times Database \rightarrow Database
\end{align}

\textbf{Semantics}:
\begin{align}
insertData(td, d) = d' \implies tableData(d') = tableData(d) \cup td 
\end{align}

\subsubsection{insertValue}
Inserts one \emph{Pair} into the \emph{TableData} in the database.
\begin{align}
insertValue: TableData \times Pair \times  Database \rightarrow Database
\end{align}

\textbf{Semantics}:
\begin{align}
insertValue(td, p, d) = d' \implies tableData(d') = tableData(d) \setminus td \cup td' \land pairs(td') = pairs(td) \cup p 
\end{align}


\subsubsection{isReferenced}
Returns true if there is a foreign key, which references given class, in the database.
\begin{align}
& notReferenced: TableSchema \times Database \rightarrow Boolean \\
& notReferenced(t, d) = \begin{cases}
 false \\
 \mathbf{if} \; \forall t_s \in tableSchemas(d), fk \in foreignKeys(t_s) : reference(fk) \neq t \\\\
 true \\
 \mathbf{if} \;  \exists t_s \in tableSchemas(d), fk \in foreignKeys(t_s) : reference(fk) = t
 \end{cases}
\end{align}

\subsubsection{reference}
Returns a sequence of tableSchemas, which contains reference to a given tableSchema.
\begin{align}
& reference: TableData \times Database \rightarrow TableData \\
& reference(td, d) = e \\
& \;\; e \in tableSchemas(d) \land \exists f \in pairsForeignKeyValue(pairs(e)) \land reference(f) = td  
\end{align}


\subsubsection{valueOfColumn}
Returns the value of given column from set of pairs.
\begin{align}
& valueOfColumn: Column \times Pair \rightarrow Value \\
& valueOfColumn(c, p) = value(p) \land c = column(p)
\end{align}

\subsubsection{pairOfColumn}
Returns the \emph{Pair} in \emph{TableData} which reference given column.
\begin{align}
& pairOfColumn: Column \times TableData \rightarrow Pair \\
& pairOfColumn(c, td) = p \implies p \in pairs(td) \land column(p) = c
\end{align}


\subsubsection{pairsColumValue}
Returns all \emph{Pairs} which references a column.
\begin{align}
& pairsColumValue: Pair \rightarrow ColumnValue \\
& pairsColumValue(p) = q \implies q \in p \land column(q) \neq \emptyset  
\end{align}

\subsubsection{pairsForeignKeyValue}
Returns all \emph{Pairs} which references a foreign key.
\begin{align}
& pairsForeignKeyValue : Pair \rightarrow ForeignKeyValue \\
& pairsForeignKeyValue(p) = q \implies q \in p \land reference(q) \neq \emptyset 
\end{align}


\subsubsection{copyPropertyAsTable}
The transformation copies the property which is represented as a table in the database.
\begin{align}
copyPropertyAsTable: TableSchema \times TableSchema \times Mapping \times Database \rightarrow Database
\end{align}
\begin{align}
& copyPropertyAsTable(ts_s, ts_t, m, d) = d' \implies \nonumber  \\
& \;\;\; fk_1 = foreignKeys(ts_s) \land fk_2 = foreignKey(ts_t, constraints(fk) \land \nonumber   \\
& \;\;\;  \exists ts_2 = (l, primaryKey(ts), columns(ts), fk_2) :  \nonumber \\
& \;\;\;  ts_2 \in tableSchemas(d') \land ts_2 \notin tableSchemas(d) \nonumber \\
& \;\;\; \forall q \in ran(m) : p \in dom(m) \land m(m) = q \land \nonumber \\  
& \;\;\; insertData(next(sequence(primaryKey(ts_2))), pairsColumValue(p) \cup (fk_2, id(q)) , d')  \nonumber \\
& \;\;\; \mathbf{if} \; \forall t \in tableSchemas(d) : label(t) \neq l \land ts \in tableSchemas(d)
\end{align}

\subsubsection{propToClass}
The transformation creates a class from the given property.
\begin{align}
& propToClass: Property \times Label \rightarrow Class \\
& propToClass(p, l) = class(l, p, \emptyset, OBJECT)
\end{align}

\subsubsection{childern}
Returns all children of the given class.
\begin{align}
& childern : Class \times Application \rightarrow Class \\
& childern(c, a) = e \implies class(inheritance(e)) = c \land c \in classes(a) \land e \in classes(a)
\end{align}



\end{document}

\subsubsection{}
\begin{align}
\end{align}

\textbf{Semantics}:
\begin{align}
\end{align}


\begin{align}
\end{align}
\begin{align}
\Psi
\end{align}

\begin{align}
\Phi
\end{align}


%TODO is it needed?
\subsubsection{copyColumnStructure}
\begin{align}
& copyColumnStructure: TableSchema \times TableSchema \times Column \times Label \times Database  \nonumber \\
& \;\;\; \rightarrow Database
\end{align}

\textbf{Semantics}:
\begin{align}
copyColumnStructure(ts_1, ts_2, col, l, d) = d' \implies 
\end{align}

%TODO is it needed?
\subsubsection{copyTableStructure}
\begin{align}
copyTableStructure: TableSchema \times Label \times Database \rightarrow Database
\end{align}

\textbf{Semantics}:
\begin{align}
& copyTableStructure(ts, l d) = d' \implies \exists ts_n \in tableSchemas(d') : ts_n \notin tableSchemas(d)   \nonumber \\
& \;\;\; \land label(ts_n) = l \land \forall col \in coluumns(ts) \exists col_n \in columns(ts_n)
\end{align}

